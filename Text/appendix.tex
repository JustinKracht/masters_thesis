\clearpage
\makeatletter
\efloat@restorefloats
\makeatother


\begin{appendix}
\hypertarget{regression-diagnostics}{%
\section{Regression Diagnostics}\label{regression-diagnostics}}

\hypertarget{models-1a-and-1b-regression-models-predicting-log-textrmd_smathbfr_textrmsm-mathbfr_textrmpop}{%
\subsection{\texorpdfstring{Models 1A and 1B: Regression models
predicting
\(\log \textrm{D}_s(\mathbf{R}_{\textrm{Sm}}, \mathbf{R}_{\textrm{Pop}})\)}{Models 1A and 1B: Regression models predicting \textbackslash log \textbackslash textrm\{D\}\_s(\textbackslash mathbf\{R\}\_\{\textbackslash textrm\{Sm\}\}, \textbackslash mathbf\{R\}\_\{\textbackslash textrm\{Pop\}\})}}\label{models-1a-and-1b-regression-models-predicting-log-textrmd_smathbfr_textrmsm-mathbfr_textrmpop}}

Models 1A and 1B were a linear mixed-effects models predicting the (log)
scaled distance between the smoothed and model-implied population
correlation matrix and was fit using the R \emph{lme4} package (Version
1.1.23; Bates, Mächler, Bolker, \& Walker, 2015). Model 1A was a linear
model fit using all simulation variables and their interactions. In
Model 1B, second-degree polynomial terms were added for number of
factors, number of subjects per item, factor loading, and model error.
Diagnostic plots showing standardized residuals plotted against fitted
values for both models, quantile-quantile (QQ) plots of the residuals,
and QQ plots for the random intercept terms are shown in Figures
@ref(fig:residuals-vs-fitted-RpopRsm), @ref(fig:qq-plot-RpopRsm), and
@ref(fig:qq-plot-randInt-RpopRsm) respectively. These plots show that
some assumptions of the linear mixed-effects model seem to have been
violated for Models 1A and 1B, even after applying a log-transformation
to the response variable.

\begin{figure}

{\centering \includegraphics[width=7.54in]{/home/justin/Documents/masters_thesis/Text/figs/RpopRsm_resid_qq} 

}

\caption{Quantile-quantile plot of residuals for Models 1A and 1B.}(\#fig:qq-plot-RpopRsm)
\end{figure}

\begin{figure}

{\centering \includegraphics[width=7.54in]{/home/justin/Documents/masters_thesis/Text/figs/RpopRsm_ranef_qq} 

}

\caption{Quantile-quantile plot of random intercept terms for Models 1A and 1B.}(\#fig:qq-plot-randInt-RpopRsm)
\end{figure}

Figure @ref(fig:residuals-vs-fitted-RpopRsm) shows that the variance of
the residuals was not constant over the range of fitted values for both
the linear and polynomial models. In particular, for both models there
was little variation near the edges of the range of fitted values and a
large amount of variation near the center of the distribution of fitted
values. Therefore, the homoscedasticity assumption seemed to have been
violated. Moreover, Figure @ref(fig:qq-plot-RpopRsm) shows that the
assumption of normally-distributed errors was also likely violated. In
particular, Figure @ref(fig:qq-plot-RpopRsm) shows that the
distributions of residuals (for both models) had heavy tails and had a
slight positive skew (Model 1A: kurtosis = 16.25, skew = 0.60; Model 1B:
kurtosis = 18.61, skew = 0.23). Finally, Figure
@ref(fig:qq-plot-randInt-RpopRsm) shows that the random effects (random
intercepts) were not normally-distributed for either the linear or
polynomial model (Model 1A: kurtosis = 5.52, skew = 1.52; Model 1B:
kurtosis = 10.33, skew = 0.59). To address these violations of the model
assumptions, I first attempted to fit a robust mixed-effects model using
\texttt{rlmer()} function in the R \emph{robustlmm} package (Version
2.3; Koller, 2016). Unfortunately, the data set was too large for the
\texttt{rlmer()} function to handle. I also tried a more complex
transformation of the dependent variable (using a Box-Cox power
transformation; Box \& Cox, 1964), but it produced no discernible
benefit compared to a log transformation.

\begin{figure}

{\centering \includegraphics[width=7.54in]{/home/justin/Documents/masters_thesis/Text/figs/RpopRsm_resid_vs_fitted} 

}

\caption{Residuals plotted against fitted values for Models 1A and 1B.}(\#fig:residuals-vs-fitted-RpopRsm)
\end{figure}

The apparent violations of the assumptions of the mixed-effects model
were concerning. However, inference for the fixed effects in
mixed-effects models seems to be somewhat robust to these violations. In
particular, Jacqmin-Gadda, Sibillot, Proust, Molina, \& Thiébaut (2007)
showed that inference for fixed effects is robust for non-Gaussian and
heteroscedastic errors. Moreover, Jacqmin-Gadda et al.~(2007) cited
several studies indicating that inference for fixed effects is also
robust to non-Gaussian random effects (Butler \& Louis, 1992; Verbeke \&
Lesaffre, 1997; Zhang \& Davidian, 2001). Finally, the purpose of the
present analysis was to obtain estimates of the fixed effects of matrix
smoothing methods (and the interactions between smoothing methods and
the other design factors) on population correlation matrix recovery.
Neither \(p\)-values nor confidence intervals were of primary concern.
Therefore, the apparent violation of some model assumptions likely did
not affect the main results of this study.

\hypertarget{models-2a-and-2b-regression-models-predicting-log-textrmrmsemathbff-hatmathbff}{%
\subsection{\texorpdfstring{Models 2A and 2B: Regression models
predicting
\(\log \textrm{RMSE}(\mathbf{F}, \hat{\mathbf{F}})\)}{Models 2A and 2B: Regression models predicting \textbackslash log \textbackslash textrm\{RMSE\}(\textbackslash mathbf\{F\}, \textbackslash hat\{\textbackslash mathbf\{F\}\})}}\label{models-2a-and-2b-regression-models-predicting-log-textrmrmsemathbff-hatmathbff}}

Models 2A and 2B were mixed-effects models predicting
\(\log \textrm{RMSE}(\mathbf{F}, \hat{\mathbf{F}})\) and fit using the R
\emph{lme4} package (Bates, Mächler, Bolker, \& Walker, 2015). Model 2A
was a linear model fit using all simulation variables and their
interactions. In Model 2B, second-degree polynomial terms were added for
number of factors, number of subjects per item, factor loading, and
model error. As with Models 1A and 1B, diagnostic plots showing
standardized residuals plotted against fitted values for both models, QQ
plots for the residuals, and QQ plots for the random intercept terms are
shown in Figures @ref(fig:residuals-vs-fitted-RpopRsm),
@ref(fig:qq-plot-loading-recovery), and
@ref(fig:qq-plot-randInt-loadings) respectively.

\begin{figure}

{\centering \includegraphics[width=7.54in]{/home/justin/Documents/masters_thesis/Text/figs/loading_resid_vs_fitted} 

}

\caption{Residuals plotted against fitted values for Models 2A and 2B.}(\#fig:residuals-vs-fitted-loading-recovery)
\end{figure}

\begin{figure}

{\centering \includegraphics[width=7.54in]{/home/justin/Documents/masters_thesis/Text/figs/loading_resid_qq} 

}

\caption{Quantile-quantile plot of residuals for Models 2A and 2B.}(\#fig:qq-plot-loading-recovery)
\end{figure}

\begin{figure}

{\centering \includegraphics[width=7.54in]{/home/justin/Documents/masters_thesis/Text/figs/loading_ranef_qq} 

}

\caption{Quantile-quantile plot of random intercept terms for Models 2A and 2B.}(\#fig:qq-plot-randInt-loadings)
\end{figure}

These plots indicate many of the same issues in Models 2A and 2B as were
seen for Models 1A and 1B. First, Figure
@ref(fig:residuals-vs-fitted-RpopRsm) shows clear evidence of
non-homogeneous conditional error variance for both the linear and
polynomial models. Specifically, the residual variance seemed generally
to be larger for larger fitted values. Second, Figure
@ref(fig:qq-plot-loading-recovery) shows that the distribution of
residuals for both models was non-normal and similar to the
distributions of the residuals from Model 1A and 1B (i.e.,
positively-skewed and having heavy tails). Finally, Figure
@ref(fig:qq-plot-randInt-loadings) shows that the estimated random
effects were likewise not normally-distributed. The distribution of
random intercepts was positively-skewed with heavy tails. Alternative
transformations of the dependent variable were tried but did not seem to
improve model fit compared to a log transformation. As with Model 1,
these violations of the model assumptions are somewhat concerning and
indicate that the estimated parameters---the estimated standard errors,
in particular---should be treated with some degree of skepticism.
However, the main results of the study are unlikely to have been
affected greatly by these violations of the model assumptions.
\end{appendix}
