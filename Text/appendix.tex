\clearpage
\makeatletter
\efloat@restorefloats
\makeatother


\begin{appendix}
\hypertarget{regression-diagnostics}{%
\section{Regression Diagnostics}\label{regression-diagnostics}}

\hypertarget{model-1-regression-model-predicting-log-textrmd_smathbfr_textrmpop-mathbfr_textrmsm}{%
\subsection{\texorpdfstring{Model 1: Regression model predicting
\(\log \textrm{D}_s(\mathbf{R}_{\textrm{Pop}}, \mathbf{R}_{\textrm{Sm}})\)}{Model 1: Regression model predicting \textbackslash log \textbackslash textrm\{D\}\_s(\textbackslash mathbf\{R\}\_\{\textbackslash textrm\{Pop\}\}, \textbackslash mathbf\{R\}\_\{\textbackslash textrm\{Sm\}\})}}\label{model-1-regression-model-predicting-log-textrmd_smathbfr_textrmpop-mathbfr_textrmsm}}

Model 1 was a linear mixed-effects model predicting the (log) scaled
distance between the smoothed and model-implied population correlation
matrix and was fit using the R \emph{lme4} package (Version 1.1.23;
Bates, Mächler, Bolker, \& Walker, 2015). Diagnostic plots showing
standardized residuals plotted against fitted values for the model, a
quantile-quantile plot for the residuals, and a quantile-quantile plot
for the random intercept terms are shown in Figures
@ref(fig:residuals-vs-fitted-RpopRsm), @ref(fig:qq-plot-RpopRsm), and
@ref(fig:qq-plot-randInt-RpopRsm) respectively. These plots showed that
some assumptions of the linear mixed-effects model seemed not to be
reasonable for Model 1, even after using a log-transformation on the
response variable.

First, Figure @ref(fig:residuals-vs-fitted-RpopRsm) showed that the
variance of the residuals was not constant over the range of fitted
values. In particular, there was little variation near the edges of the
range of fitted values and a large amount of variation near the center
of the distribution of fitted values. Therefore, the homoscedasticity
assumption of the linear mixed-effects model seemed to have been
violated. Moreover, Figure @ref(fig:qq-plot-RpopRsm) showed that the
assumption of normally-distributed errors also seemed likely to have
been violated. In particular, Figure @ref(fig:qq-plot-RpopRsm) showed
that the distribution of residuals had heavy tails and was positively
skewed. Finally, Figure @ref(fig:qq-plot-randInt-RpopRsm) shows that the
random effects (random intercepts) were also not normally distributed
--- the distribution was positively skewed. To address these violations
of the model assumptions, I first attempted to fit a robust linear
mixed-effects model using \texttt{rlmer} function in the R
\emph{robustlmm} package (Version 2.3; Koller, 2016). Unfortunately, the
data were too large for the \texttt{rlmer} function to handle. I also
tried a more complex transformation of the dependent variable (using a
Box-Cox power transformation; Box \& Cox, 1964), but it produced no
discernable benefit compared to a log transformation.

The apparent violations of the assumptions of the linear mixed-effects
model were concerning. However, inference for the fixed effects in
mixed-effects models seems to be somewhat robust to these violations. In
particular, Jacqmin-Gadda, Sibillot, Proust, Molina, \& Thiébaut (2007)
showed that inference for fixed effects was robust for non-Gaussian and
heteroscedastic errors. Moreover, Jacqmin-Gadda et al.~(2007) cite
several studies that indicate that inference for fixed effects are also
robust to non-Gaussian random effects (Butler \& Louis, 1992; Verbeke \&
Lesaffre, 1997; Zhang \& Davidian, 2001). Finally, the purpose of this
analysis was to obtain estimates of the fixed effects of matrix
smoothing methods (and the interactions between smoothing methods and
the other design factors) on population correlation matrix recovery.
Neither \(p\)-values nor confidence intervals were of primary concern.
Therefore, the apparent violation of some model assumptions likely did
not affect the main results of the study.

\begin{figure}

{\centering \includegraphics[width=7in]{/Volumes/GoogleDrive/My Drive/School/masters_thesis/Text/figs/RpopRsm_resid_vs_fitted} 

}

\caption{Standardized residuals plotted against fitted values for Model 1.}(\#fig:residuals-vs-fitted-RpopRsm)
\end{figure}

\begin{figure}

{\centering \includegraphics[width=5.6in]{/Volumes/GoogleDrive/My Drive/School/masters_thesis/Text/figs/RpopRsm_resid_qq} 

}

\caption{Quantile-quantile plot of residuals for Model 1.}(\#fig:qq-plot-RpopRsm)
\end{figure}

\begin{figure}

{\centering \includegraphics[width=5.6in]{/Volumes/GoogleDrive/My Drive/School/masters_thesis/Text/figs/RpopRsm_ranef_qq} 

}

\caption{Quantile-quantile plot of random intercept terms for Model 1.}(\#fig:qq-plot-randInt-RpopRsm)
\end{figure}

\hypertarget{model-2-regression-model-predicting-log-textrmrmsemathbflambda-hatmathbflambda}{%
\subsection{\texorpdfstring{Model 2: Regression model predicting
\(\log \textrm{RMSE}(\mathbf{\Lambda}, \hat{\mathbf{\Lambda}})\)}{Model 2: Regression model predicting \textbackslash log \textbackslash textrm\{RMSE\}(\textbackslash mathbf\{\textbackslash Lambda\}, \textbackslash hat\{\textbackslash mathbf\{\textbackslash Lambda\}\})}}\label{model-2-regression-model-predicting-log-textrmrmsemathbflambda-hatmathbflambda}}

Model 2 was a linear mixed-effects model predicting
\(\log \textrm{RMSE}(\mathbf{\Lambda}, \hat{\mathbf{\Lambda}})\) fit
using the R \emph{lme4} package (Bates, Mächler, Bolker, \& Walker,
2015). As with Model 1, diagnostic plots showing standardized residuals
plotted against fitted values for the model, a quantile-quantile plot
for the residuals, and a quantile-quantile plot for the random intercept
terms are shown in Figures @ref(fig:residuals-vs-fitted-RpopRsm),
@ref(fig:qq-plot-loading-recovery), and
@ref(fig:qq-plot-randInt-loadings) respectively. These plots indicate
many of the same issues in Model 2 as were seen for Model 1. First,
Figure @ref(fig:residuals-vs-fitted-RpopRsm) shows clear evidence of
non-homogenous conditional error variance. The residual variance seemed
generally to be larger for larger fitted values. Second, Figure
@ref(fig:qq-plot-loading-recovery) showed that the distribution of
residuals for Model 2 was non-normal and similar to the distribution of
the residuals from Model 1 (i.e., positively-skewed and having heavy
tails). Finally, Figure @ref(fig:qq-plot-randInt-loadings) showed that
the estimated random effects were also not normally-distributed (similar
to Model 1). The distribution of random intercepts was positively-skewed
with heavy tails. Alternative transformations of the dependent variable
were tried but did not seem to improve model fit compared to a log
transformation. As with Model 1, these violations of the model
assumptions are somewhat concerning and indicate that the estimated
parameters---the estimated standard errors, in particular---should be
treated with some degree of skepticism. However, the main results of the
study are unlikely to have been affected greatly by these violations of
the model assumptions.

\begin{figure}

{\centering \includegraphics[width=7in]{/Volumes/GoogleDrive/My Drive/School/masters_thesis/Text/figs/loading_resid_vs_fitted} 

}

\caption{Standardized residuals plotted against fitted values for Model 2.}(\#fig:residuals-vs-fitted-loading-recovery)
\end{figure}

\begin{figure}

{\centering \includegraphics[width=5.6in]{/Volumes/GoogleDrive/My Drive/School/masters_thesis/Text/figs/loading_resid_qq} 

}

\caption{Quantile-quantile plot of residuals for Model 2.}(\#fig:qq-plot-loading-recovery)
\end{figure}

\begin{figure}

{\centering \includegraphics[width=5.6in]{/Volumes/GoogleDrive/My Drive/School/masters_thesis/Text/figs/loading_ranef_qq} 

}

\caption{Quantile-quantile plot of random intercept terms for Model 2.}(\#fig:qq-plot-randInt-loadings)
\end{figure}
\end{appendix}
